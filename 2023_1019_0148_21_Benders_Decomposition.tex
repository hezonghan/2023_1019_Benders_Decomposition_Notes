
\documentclass[twocolumn]{ctexart}
% \documentclass[onecolumn]{ctexart}

\usepackage{geometry}
\geometry{a4paper,left=1cm,right=1cm,top=1.4cm,bottom=1cm}

\usepackage{amsmath,amssymb,amsfonts}
\usepackage{xcolor}
\usepackage{graphicx}  % for \resizebox
\usepackage{arydshln}



\usepackage{hyperref}
\hypersetup{
    pdfstartview={XYZ null null 1.25},  % XYZ <left> <top> <zoom> : Sets a coordinate and a zoom factor. If anyone is null, the source link value is used.
    % pdfstartview={FitH  null},           % FitH <top> : fit the width of the page to the window
    pdfpagelayout=OneColumn,  % SinglePage/OneColumn/...
    % 
    bookmarks=true,
    bookmarksnumbered=true,
    bookmarksopen=true,  % whether bookmark tree is expanded when open
    pdfpagemode=UseOutlines,  % UseOutlines: show bookmarks when open
    % 
    linkbordercolor=blue,
}



\begin{document}

\leftline{2023\_1019\_0148\_21}

% IIIIIIIIIIIIIIIIIIIIIIIIIIIIIIIIIIIIIIIIIIIIIIIIIIIIIIIIIIIIIIIIIIIIIIIIIIIIIIIIIIIIIIIIIIIIIIIIIIII
\section{LP and Duality}

% =================================================================
\subsection{Standard Form of LP}

Any unbounded variable $x$ can be replaced into a pair of non-negative variables $(u,v)$.
\begin{equation}
    \begin{aligned}
        % \label{eq:}
        \begin{cases}
            x = u - v \\
            u \geq 0  \\
            v \geq 0  \\
        \end{cases}
    \end{aligned}
\end{equation}

Any in-equality constraint can be converted into an equality constraint,
by introducing an additional assistant non-negative variable $x'$.
\begin{equation}
    \begin{aligned}
        \label{eq:}
        \mathbf{a}_i^T  \mathbf{x}  \geq  b_i
        & \qquad \Leftrightarrow \qquad
        \begin{cases}
            \mathbf{a}_i^T \mathbf{x}  -  x'  &=  b_i \\
            x' &\geq 0 \\
        \end{cases}
        \\
        & \qquad \Leftrightarrow \qquad
        \begin{cases}
            \begin{bmatrix} \mathbf{a}_i^T &  (-1) \end{bmatrix}
            \begin{bmatrix} \mathbf{x}     \\ x' \end{bmatrix}
            &= b_i
            \\
            x' &\geq 0 \\
        \end{cases}
    \end{aligned}
\end{equation}

So we can safely represent any LP problem in its standard form,
with only non-negative variables and only equality constraints.


% =================================================================
\subsection{Definition of Dual}

Assume
$\mathbf{A} \in \mathbb{R}^{m \times n}$
,
$\mathbf{b} \in \mathbb{R}^{m \times 1}$
,
$\mathbf{c} \in \mathbb{R}^{n \times 1}$
.
% A LP problem
The Linear Programming (LP) problem
\begin{equation}
    \begin{aligned}
        % \label{eq:}
        \text{Minimize} \qquad &
            \mathbf{c}^T  \mathbf{\color{red} x}
        \\
        \text{s.t.} \qquad &
            % &  A  \mathbf{x}  &  \geq  \mathbf{b}  \\
            % &     \mathbf{x}  &   \geq  \mathbf{0}  \\
            \begin{cases}
                \mathbf{A}  \mathbf{\color{red} x}  &  \geq  \mathbf{b}  {\color{gray} \quad \in \mathbb{R}^{m \times 1}}  \\
                            \mathbf{\color{red} x}  &  \geq  \mathbf{0}  {\color{gray} \quad \in \mathbb{R}^{n \times 1}}  \\
            \end{cases}
    \end{aligned}
\end{equation}
has a dual problem
\begin{equation}
    \begin{aligned}
        % \label{eq:}
        \text{Maximize} \qquad &
            \mathbf{b}^T  \mathbf{\color{red} \lambda}
        \\
        \text{s.t.} \qquad &
            \begin{cases}
                \mathbf{A}^T  \mathbf{\color{red} \lambda}  &  \leq  \mathbf{c}  {\color{gray} \quad \in \mathbb{R}^{n \times 1}}  \\
                              \mathbf{\color{red} \lambda}  &  \geq  \mathbf{0}  {\color{gray} \quad \in \mathbb{R}^{m \times 1}}  \\
            \end{cases}
    \end{aligned}
\end{equation}



\subsection{The Duality Theorem}




\begin{equation}
    \begin{aligned}
        \label{eq:}
    \end{aligned}
\end{equation}






\end{document}

